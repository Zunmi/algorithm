%
% 6.006 problem set 3 solutions template
%
\documentclass[12pt,twoside]{article}

\input{macros-sp20}
\newcommand{\theproblemsetnum}{3}

\title{6.006 Problem Set \theproblemsetnum}

\begin{document}

\handout{Problem Set \theproblemsetnum}{July 11,2025}

\setlength{\parindent}{0pt}
\medskip\hrulefill\medskip

{\bf Name:} Zunmi

\medskip

{\bf Collaborators:} None

\medskip\hrulefill

%%%%%%%%%%%%%%%%%%%%%%%%%%%%%%%%%%%%%%%%%%%%%%%%%%%%%
% See below for common and useful latex constructs. %
%%%%%%%%%%%%%%%%%%%%%%%%%%%%%%%%%%%%%%%%%%%%%%%%%%%%%

% Some useful commands:
%$f(x) = \Theta(x)$
%$T(x, y) \leq \log(x) + 2^y + \binom{2n}{n}$
% {\tt code\_function}


% You can create unnumbered lists as follows:
%\begin{itemize}
%    \item First item in a list
%        \begin{itemize}
%            \item First item in a list
%                \begin{itemize}
%                    \item First item in a list
%                    \item Second item in a list
%                \end{itemize}
%            \item Second item in a list
%        \end{itemize}
%    \item Second item in a list
%\end{itemize}

% You can create numbered lists as follows:
%\begin{enumerate}
%    \item First item in a list
%    \item Second item in a list
%    \item Third item in a list
%\end{enumerate}

% You can write aligned equations as follows:
%\begin{align}
%    \begin{split}
%        (x+y)^3 &= (x+y)^2(x+y) \\
%                &= (x^2+2xy+y^2)(x+y) \\
%                &= (x^3+2x^2y+xy^2) + (x^2y+2xy^2+y^3) \\
%                &= x^3+3x^2y+3xy^2+y^3
%    \end{split}
%\end{align}

% You can create grids/matrices as follows:
%\begin{align}
%    A =
%    \begin{bmatrix}
%        A_{11} & A_{21} \\
%        A_{21} & A_{22}
%    \end{bmatrix}
%\end{align}

% You can include images and PDFs as follows:
% \includegraphics[width=0.5\textwidth]{img.jpg}

\begin{problems}

\problem  % Problem 1


\begin{problemparts}
\problempart % Problem 1a
\begin{verbatim}
    0:[36,92]
    1:
    2:
    3:
    4:[56]
    5:[47,61,33]
    6:[52]
\end{verbatim}
\problempart % Problem 1b
$c = 13$
\end{problemparts}

\newpage

\problem  % Problem 2

\begin{problemparts}
\problempart % Problem 2a
guarantee:$k_1 = 1,k_2 = n+1$
\problempart % Problem 2b
guarantee:choose $k_1,k_2$small enough so that $k_in//u = 0$
\problempart % Problem 2c
the highest probability is $\frac{1}{n}$
\end{problemparts}

\newpage

\problem  % Problem 3

\begin{problemparts}
\problempart % Problem 3a
word can be upper bounded by $128^{16logn * 8}< n^{33}$,so radix sort. 
\problempart % Problem 3b
direct access array of size $10^{5}$
\problempart % Problem 3c
Multiplying by $n^3$,so they are m in $[0,4n^3]$,using radix sort.
\problempart % Problem 3d
merge sort using only comparisons,so $O(nlogn)$.
\end{problemparts}

\newpage

\problem  % Problem 4

\begin{problemparts}
\problempart % Problem 4a
build a hash table H,insert $b_i$ into H mapped to i,for every $b_i$
,check if $r-b_i$exists in H in expected O(1) time,then check whether it is close.
\problempart % Problem 4b
Replace each $b_i$,with tuple $(b_i,i)$,scan the B and remove all $(b_i,i)$ that $b_i > r$,
then use redix sort to an array A.\par
use two-pointer algorithm,initialize $i=0,j=|A|-1$,\par
if A[i][0] + A[j][0] < r,then $i++$;\par
if A[i][0] + A[j][0] > r,then $j--$;\par
if A[i][0] + A[j][0] = r,then return True.
\end{problemparts}

\newpage

\problem  % Problem 5

\begin{problemparts}
\problempart % Problem 5a

\problempart % Problem 5b
\problempart code in .python file
\end{problemparts}

\end{problems}

\end{document}
