%
% 6.006 problem set 1 solutions template
%
\documentclass[12pt,twoside]{article}

\input{macros-sp20}
\newcommand{\theproblemsetnum}{1}

\title{6.006 Problem Set 1}

\begin{document}

\handout{Problem Set \theproblemsetnum}{July 1,2025}

\setlength{\parindent}{0pt}
\medskip\hrulefill\medskip

{\bf Name:} Zunmi

\medskip

{\bf Collaborators:} None.

\medskip\hrulefill

%%%%%%%%%%%%%%%%%%%%%%%%%%%%%%%%%%%%%%%%%%%%%%%%%%%%%
% See below for common and useful latex constructs. %
%%%%%%%%%%%%%%%%%%%%%%%%%%%%%%%%%%%%%%%%%%%%%%%%%%%%%

% Some useful commands:
%$f(x) = \Theta(x)$
%$T(x, y) \leq \log(x) + 2^y + \binom{2n}{n}$
% {\tt code\_function}


% You can create unnumbered lists as follows:
%\begin{itemize}
%    \item First item in a list
%        \begin{itemize}
%            \item First item in a list
%                \begin{itemize}
%                    \item First item in a list
%                    \item Second item in a list
%                \end{itemize}
%            \item Second item in a list
%        \end{itemize}
%    \item Second item in a list
%\end{itemize}

% You can create numbered lists as follows:
%\begin{enumerate}
%    \item First item in a list
%    \item Second item in a list
%    \item Third item in a list
%\end{enumerate}

% You can write aligned equations as follows:
%\begin{align}
%    \begin{split}
%        (x+y)^3 &= (x+y)^2(x+y) \\
%                &= (x^2+2xy+y^2)(x+y) \\
%                &= (x^3+2x^2y+xy^2) + (x^2y+2xy^2+y^3) \\
%                &= x^3+3x^2y+3xy^2+y^3
%    \end{split}
%\end{align}

% You can create grids/matrices as follows:
%\begin{align}
%    A =
%    \begin{bmatrix}
%        A_{11} & A_{21} \\
%        A_{21} & A_{22}
%    \end{bmatrix}
%\end{align}

% You can include images and PDFs as follows:
% \includegraphics[width=0.5\textwidth]{img.jpg}

\begin{problems}

\problem  % Problem 1

\begin{problemparts}
\problempart % Problem 1a
(\(f_5,f_3,f_4,f_1,f_2\))
\problempart % Problem 1b
(\(f_1,f_2,f_5,f_4,f_3\))
\problempart % Problem 1c
(\(\{f_2,f_5\},f_4,f_1,f_3\))
\problempart % Problem 1d
(\(f_5,f_2,f_1,f_3,f_4\))
\end{problemparts}

\newpage
\problem  % Problem 2

\begin{problemparts}
\problempart % Problem 2a
\begin{verbatim}
    def reverse(D,i,k):
        if k < 2:
        return
        x1 = D.delete_at(i+k-1)
        x2 = D.delete_at(i)
        D.insert_at(i,x1)
        D.insert_at(i+k-1,x2)
        reverse(D,i+1,k-2)
    \end{verbatim}
\problempart % Problem 2b
\begin{verbatim}
    def move(D,i,k,j):
        if k < 1:
            return
        x = D.delete_at(i)
        if i > j:
            j = j-1
        D.insert_at(j,x)
        j = j+1
        if i < j:
           i = i+1
        move(D,i,k,j)

\end{verbatim}
\end{problemparts}

\newpage
\problem  % Problem 3
\begin{verbatim}
    1.build a static array of size n
    2.A,B store the marked index
\end{verbatim}
\newpage
\problem  % Problem 4

\begin{problemparts}
\problempart % Problem 4a
code in .python file
\problempart % Problem 4b
\begin{verbatim}
   code in .python file
\end{verbatim}
\problempart % Problem 4c
code in .python file
\problempart % Problem 4d 
code in .python file
\end{problemparts}

\end{problems}

\end{document}
