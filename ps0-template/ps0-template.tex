%
% 6.006 problem set 0 solutions template
%
\documentclass[12pt,twoside]{article}

\input{macros-sp20}
\newcommand{\theproblemsetnum}{0}

\title{6.006 Problem Set 0}

\begin{document}

\handout{Problem Set \theproblemsetnum}{July 2025}

\setlength{\parindent}{0pt}
\medskip\hrulefill\medskip

{\bf Name:} Zunmi

\medskip\hrulefill

%%%%%%%%%%%%%%%%%%%%%%%%%%%%%%%%%%%%%%%%%%%%%%%%%%%%%
% See below for common and useful latex constructs. %
%%%%%%%%%%%%%%%%%%%%%%%%%%%%%%%%%%%%%%%%%%%%%%%%%%%%%

% Some useful commands:
% $f(x) = \Theta(x)$
% $T(x, y) \leq \log(x) + 2^y + \binom{2n}{n}$
% \ttt{code\_function}


% You can create unnumbered lists as follows:
% \begin{itemize}
%     \item First item in a list
%         \begin{itemize}
%             \item First item in a list
%                 \begin{itemize}
%                     \item First item in a list
%                     \item Second item in a list
%                 \end{itemize}
%             \item Second item in a list
%         \end{itemize}
%     \item Second item in a list
% \end{itemize}

% You can create numbered lists as follows:
% \begin{enumerate}
%     \item First item in a list
%     \item Second item in a list
%     \item Third item in a list
% \end{enumerate}

% You can write aligned equations as follows:
% \begin{align}
%     \begin{split}
%         (x+y)^3 &= (x+y)^2(x+y) \\
%                 &= (x^2+2xy+y^2)(x+y) \\
%                 &= (x^3+2x^2y+xy^2) + (x^2y+2xy^2+y^3) \\
%                 &= x^3+3x^2y+3xy^2+y^3
%     \end{split}
% \end{align}

% You can create grids/matrices as follows:
% \begin{align}
%     A =
%     \begin{bmatrix}
%         A_{11} & A_{21} \\
%         A_{21} & A_{22}
%     \end{bmatrix}
% \end{align}

\begin{problems}

\problem  % Problem 1

\begin{problemparts}
\problempart % Problem 1a
{6,12}
\problempart % Problem 1b
7
\problempart % Problem 1c
3
\end{problemparts}

\problem  % Problem 2

\begin{problemparts}
\problempart % Problem 2a
1.5
\problempart % Problem 2b
12.25
\problempart % Problem 2c
13.75
\end{problemparts}

\problem  % Problem 3

\begin{problemparts}
\problempart % Problem 3a
True
\problempart % Problem 3b
False
\problempart % Problem 3c
False
\end{problemparts}

\problem  % Problem 4


\textbf{Base Case:} For $n = 1$, we have
\[
\sum_{i=1}^1 i^3 = 1^3 = 1, \quad \text{and} \quad \left( \frac{1(1+1)}{2} \right)^2 = 1.
\]
Thus, the formula holds for $n = 1$.

\medskip

\textbf{Inductive Hypothesis:} Assume that the formula holds for some $n = k$:
\[
\sum_{i=1}^k i^3 = \left( \frac{k(k+1)}{2} \right)^2.
\]

\medskip

\textbf{Inductive Step:} We must show that the formula also holds for $n = k+1$:
\[
\sum_{i=1}^{k+1} i^3 = \left( \frac{(k+1)(k+2)}{2} \right)^2.
\]
Starting from the left-hand side:
\[
\sum_{i=1}^{k+1} i^3 = \left( \sum_{i=1}^k i^3 \right) + (k+1)^3.
\]
Apply the inductive hypothesis:
\[
= \left( \frac{k(k+1)}{2} \right)^2 + (k+1)^3.
\]
Factor out $(k+1)^2$:
\[
= (k+1)^2 \left( \frac{k^2}{4} + (k+1) \right)
= (k+1)^2 \left( \frac{k^2 + 4k + 4}{4} \right)
= (k+1)^2 \left( \frac{(k+2)^2}{4} \right).
\]
Thus,
\[
\sum_{i=1}^{k+1} i^3 = \left( \frac{(k+1)(k+2)}{2} \right)^2.
\]

\medskip

\textbf{Conclusion:} By the principle of mathematical induction, the formula
\[
\sum_{i=1}^n i^3 = \left( \frac{n(n+1)}{2} \right)^2
\]
holds for all positive integers $n$. \qed
\newpage
\problem  % Problem 5
\textbf{Base Case:} For $|V| = 1$,the graph is trivially acyclic.\par
Assuming the claim is true for all graphs with $|V| \leq k$, we will show it holds for $|V| = k+1$.\par
$G$ is connected,so the aveerage degree of vertices is $\frac{2k}{k+1}<2$.so there exists a vertex $v$ with degree 1 connected to u.
removing $v$ and the edge $(u,v)$, we get a graph $G'$ with $|V| = k$,which is acyclic by the inductive hypothesis.Vertex $v$ cannot be part of any cycle in $G$ because it has degree 1, so $G$ is acyclic.


\vfill
\problem  % Problem 6
Submit your implementation to {\small\url{alg.mit.edu}}.

\begin{lstlisting}
def count_long_subarray(A):
    '''
    Input:  A     | Python Tuple of positive integers
    Output: count | number of longest increasing subarrays of A
    '''
    count = 0
    ##################
    see .python file
    ##################
    return count
\end{lstlisting}

\end{problems}

\end{document}
