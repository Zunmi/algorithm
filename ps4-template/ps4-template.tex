%
% 6.006 problem set 4 solutions template
%
\documentclass[12pt,twoside]{article}

\input{macros-sp20}
\newcommand{\theproblemsetnum}{4}

\title{6.006 Problem Set \theproblemsetnum}

\begin{document}

\handout{Problem Set \theproblemsetnum}{June 9,2025}

\setlength{\parindent}{0pt}
\medskip\hrulefill\medskip

{\bf Name:} Zunmi

\medskip

{\bf Collaborators:} None

\medskip\hrulefill

%%%%%%%%%%%%%%%%%%%%%%%%%%%%%%%%%%%%%%%%%%%%%%%%%%%%%
% See below for common and useful latex constructs. %
%%%%%%%%%%%%%%%%%%%%%%%%%%%%%%%%%%%%%%%%%%%%%%%%%%%%%

% Some useful commands:
%$f(x) = \Theta(x)$
%$T(x, y) \leq \log(x) + 2^y + \binom{2n}{n}$
% {\tt code\_function}


% You can create unnumbered lists as follows:
%\begin{itemize}
%    \item First item in a list
%        \begin{itemize}
%            \item First item in a list
%                \begin{itemize}
%                    \item First item in a list
%                    \item Second item in a list
%                \end{itemize}
%            \item Second item in a list
%        \end{itemize}
%    \item Second item in a list
%\end{itemize}

% You can create numbered lists as follows:
%\begin{enumerate}
%    \item First item in a list
%    \item Second item in a list
%    \item Third item in a list
%\end{enumerate}

% You can write aligned equations as follows:
%\begin{align}
%    \begin{split}
%        (x+y)^3 &= (x+y)^2(x+y) \\
%                &= (x^2+2xy+y^2)(x+y) \\
%                &= (x^3+2x^2y+xy^2) + (x^2y+2xy^2+y^3) \\
%                &= x^3+3x^2y+3xy^2+y^3
%    \end{split}
%\end{align}

% You can create grids/matrices as follows:
%\begin{align}
%    A =
%    \begin{bmatrix}
%        A_{11} & A_{21} \\
%        A_{21} & A_{22}
%    \end{bmatrix}
%\end{align}

% You can include images and PDFs as follows:
% \includegraphics[width=0.5\textwidth]{img.jpg}

\begin{problems}

\problem  % Problem 1

\begin{problemparts}
\problempart % Problem 1a
16(skew=2),37(skew=-2)
\problempart % Problem 1b
\includegraphics[width=0.5\textwidth]{ps4-1-b.jpg}
\problempart % Problem 1c
\includegraphics[width=0.5\textwidth]{ps4-1-c.jpg}
\end{problemparts}

\newpage
\problem  % Problem 2

\begin{problemparts}
\problempart % Problem 2a
Min-heap
\problempart % Problem 2b
Mx-heap
\problempart % Problem 2c
{[0,2,2,8,9,13]}
\problempart % Problem 2d
Min-heap
\end{problemparts}

\newpage
\problem  % Problem 3

\begin{problemparts}
\problempart % Problem 3a
Build a max-heap from A keyed on $s_i$,mapping to $r_i$,
then use $delete\_max$ k times to extract the k largest elements.
\problempart % Problem 3b
\begin{verbatim}
    greaters = []
    def greater_than_x(A,i=0, x):
        assert len(A) > 0
        while i < len(A):
            if A[i] <= x:
                greater_than_x(A, (i+1)//2, x)
            else:
                greaters.append(A[i])
                greater_than_x(A, 2*i+1, x)
                greater_than_x(A, 2*i+2, x)
        return greaters

\end{verbatim}

\end{problemparts}

\newpage
\problem  % Problem 4
1. Priority Queue P on the solar farms(a Max-Heap), storing for each solar farm its address $s_i$, capacity $c_i$,
and its available capacity $a_i$ (initially $a_i = c_i$), keyed on available capacity;\par
2. Set data structure B(Hash-Table) mapping each building’s name $b_j$ to the address of the solarpowered building’s name $bj$ to the address of the solar
farm $s_i$ that it is connected to and its demand $d_j$;\par
3. Set data structure F(Hash-Table) mapping the address of each solar farm $s_i$ to: (1) its own Set data struc-
ture $B_i$(Hash-Table) containing the buildings associated with that farm, and (2) a pointer to the location of
$s_i$ in P .\par
the operations:\par
initialize(S): build Set data structures P and then F from S, and initialize B as empty. \par
power\_on($b_j,d_j$):  First, find a solor farm to connect by deleting a solar farm si from P
having largest available capacity ci (delete\_max) and checking whether it’s capacity is at
least $d_j$ . There are two cases:\par
a. $d_j > c_i$, so reinsert the solar farm back into P (relinking a pointr from F to a location
in P ) and return that no solar farm can currently support the building.\par
b. $d_j <= c_i$, so subtract $d_j$ from $c_i$ and reinsert it back into P (relinking a pointer). Then,
add $b_j$ to B mapping to $s_i$, and then find the $B_i$ in F associated with $s_i$ and add$b_j$ to $B_i$.
\par
power off($b_j$): Lookup the $s_i$ and $d_j$ associated with $b_j$ in B, lookup $B_i$ in F using $s_i$, and
remove $b_j$ from $B_i$. Lastly, go to $s_i$’s location in P and remove $s_i$ from P , increase $c_i$ by
$d_j$,and reinsert $s_i$ into P. \par
customers($s_i$): lookup $B_i$ in F using $s_i$, and return all names stored in $B_i$. \par

\newpage
\problem  % Problem 5
Store the matrices in a Sequence AVL tree T,where every node is augmented with the following information:\par
1. the size of the subtree of node;\par
2. the ordered product of all matrices in the subtree rooted of node;\par
operations:\par
initialize($M$): build the AVL tree T from M\par
update\_joint(k,M):find the kth node v in T,replace the v.matrix with M, and update the ordered product of all matrices in the subtree rooted at v.\par
full\_transformation(): return the ordered product of root in T.\par
\newpage
\problem  % Problem 6
\begin{problemparts}
\problempart % Problem 6a
find tne maximum 
\problempart % Problem 6b
\problempart % Problem 6c
\problempart Submit your implementation to {\small\url{alg.mit.edu}}.
\end{problemparts}

\end{problems}

\end{document}
